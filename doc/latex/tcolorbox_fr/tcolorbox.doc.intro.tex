% !TeX root = tcolorbox.tex
% include file of tcolorbox.tex (manual of the LaTeX package tcolorbox)
\clearpage
\section{Introduction}%
\tcbset{external/prefix=external/intro_}%
% The package originates from the first edition of my book
% \flqq{\citetitle{sturm:latex}\frqq~\cite{sturm:latex}
% in about 2006.
% For the \LaTeX\ examples and tutorials given there, I wanted to have
% accentuated and colored boxes to display source code and
% compiled text in combination.
% Since, in my opinion, this type of boxes is also quite useful to highlight definitions
% and theorems, I applied them for my lecture notes in
% mathematics \cite{sturm:mathe1,sturm:mathe2,sturm:mathe3}
% as well.
% With this package, you are invited to apply these boxes for similar projects.

L'origine de la création de ce paquet a été l'écriture de la première édition de mon livre
\flqq~{\citetitle{sturm:latex}~\frqq~\cite{sturm:latex} vers 2006.
Dans les exemples \LaTeX\ et les tutoriels fournis dans celui-ci, je désirais utiliser des
boîtes colorées pour mettre en valeur l'affichage du code source et de la version
compilée en même temps. Comme à mon avis, ce type de boîtes est également
assez pratique pour mettre en évidence des définitions et des théorèmes, je les ai
également  utilisés dans mes cours de mathématiques
\cite{sturm:mathe1,sturm:mathe2,sturm:mathe3}.
Grâce à ce paquet, je vous invite à utiliser ces boîtes pour des projets similaires.

% The breaking news for version 2.00 was the support for breakable boxes.
% This feature allows new applications of the package without
% affecting the core package too much if you do not need boxes to break automatically.
% With version 2.20, the often requested \enquote{side by side} mode for listings has been added.
% With version 3.00, boxed titles are introduced together with improved
% customization options for overlays, underlays, finishes, and own code extensions.

La nouveauté pour la version 2.00 était la prise en charge de boîtes divisibles.
Ce comportement permit de nouveaux usages pour ce paquet sans trop affecter
le cœur du paquet même si vous n'avez pas besoin de boîtes qui se divisent automatiquement.
À partir de la version 2.20, le mode «~côte à côte~», très souvent réclamé pour l'écriture de
programme informatique a été ajouté.
À partir de la version 2.30, les titres de boîtes ont été introduits ainsi que des
améliorations des options de personnalisation des sur-couches, sous-couches, finitions et
des extensions personnelles de code.


% \begin{tcolorbox}[enhanced,boxrule=0mm,boxsep=0mm,frame hidden,interior hidden,
%   left=0mm,right=0mm,top=0mm,bottom=0mm,watermark opacity=0.25,watermark zoom=1.2,before=\par\smallskip,
%   clip watermark=false,
%   watermark tikz={%
%     \path[fill=yellow,draw=yellow!75!red] (0,0) circle (1cm);
%     \fill[red] (45:5mm) circle (1mm);
%     \fill[red] (135:5mm) circle (1mm);
%     \draw[line width=1mm,red] (215:5mm) arc (215:325:5mm);}]
% Since the first public release in 2011, I received a lot of feedback from all over the world.
% I want to thank all who wrote me for supporting this package by sending bug reports
% and ideas for new or better features.
% \end{tcolorbox}

\begin{tcolorbox}[enhanced,boxrule=0mm,boxsep=0mm,frame hidden,interior hidden,
  left=0mm,right=0mm,top=0mm,bottom=0mm,watermark opacity=0.25,watermark zoom=1.2,before=\par\smallskip,
  clip watermark=false,
  watermark tikz={%
    \path[fill=yellow,draw=yellow!75!red] (0,0) circle (1cm);
    \fill[red] (45:5mm) circle (1mm);
    \fill[red] (135:5mm) circle (1mm);
    \draw[line width=1mm,red] (215:5mm) arc (215:325:5mm);}]
Depuis la première version publique en 2011, j'ai reçu beaucoup de commentaires du monde entier.
J'aimerais remercier tous ceux qui m'ont écrit pour soutenir ce paquet en m'envoyant des
rapports d'anomalie et des idées pour de nouvelles fonctionnalités ou améliorations.
\end{tcolorbox}



\subsection{Installation}
% Typically, |tcolorbox| will be installed as part of a major \LaTeX\ distribution
% and there is nothing special to do for a user.

En général, |tcolorbox| est déjà installé dans le cadre d'une distribution
 \LaTeX\ principale et vous n'avez rien de spécial à faire en tant qu'utilisateur.

% If you intend to make a local
% installation \emph{by hand}, see the |README| file of the |tcolorbox| package
% for some hints. The short story is: you have to install not only
% |tcolorbox.sty|, but also all |*.code.tex| files in the local |texmf| tree.

Si vous désirez faire une installation locale  \emph{par vous même}, lisez le fichier
|README| du paquet |tcolorbox| pour avoir des conseils. Pour faire bref, vous devez non
seulement installer |tcolorbox.sty|, mais aussi tous les fichiers |*.code.tex| dans
l'arborescence locale |texmf|.

\subsection{Chargement du paquet}
% The base package |tcolorbox| loads the packages
% |pgf| \cite{tantau:tikz_and_pgf}, |verbatim| \cite{schoepf:2001a},
% |etoolbox| \cite{lehmann:etoolbox},
% and |environ| \cite{robertson:2014a}.
% |tcolorbox| itself is loaded in the usual manner in the preamble:
% \begin{dispListing}
% \usepackage{tcolorbox}
% \end{dispListing}
% The package takes option keys in the key-value syntax.
% Alternatively, you may use these keys later in the preamble with
% \refCom{tcbuselibrary} (see there).
% For example, the key to typeset listings is:
% \begin{dispListing}
% \usepackage[listings]{tcolorbox}
% \end{dispListing}

Le paquet de base |tcolorbox| charge les paquets |pgf| \cite{tantau:tikz_and_pgf},
|verbatim| \cite{schoepf:2001a}, |etoolbox| \cite{lehmann:etoolbox} et |environ|
\cite{robertson:2014a}.
|tcolorbox| lui-même se charge de manière classique dans le préambule de la manière suivante~:
\begin{dispListing}
\usepackage{tcolorbox}
\end{dispListing}
Le paquet possède des options et utilise une syntaxe de type clé-valeur.
Vous pouvez également utiliser ces options plus tard dans le préambule en utilisant
\refCom{tcbuselibrary} (voir ici).
Par exemple, pour utiliser l'option |listings|, la syntaxe est~:
\begin{dispListing}
\usepackage[listings]{tcolorbox}
\end{dispListing}



\clearpage
\subsection{Bibliothèques}\label{sec:bibliothek}
% The base package |tcolorbox| is extendable by program libraries.
% This is done by using option keys while loading the package or inside
% the preamble by applying the following macro with the same set of keys.

% \begin{docCommand}{tcbuselibrary}{\marg{key list}}
%   Loads the libraries given by the \meta{key list}.
% \begin{dispListing}
% \tcbuselibrary{listings,theorems}
% \end{dispListing}
% \end{docCommand}

Le paquet de base |tcolorbox| est extensible grâce des bibliothèques de programmes.
Pour cela, il suffit de lister les clés d'option au moment du chargement du paquet
ou, dans le préambule, en appelant la macro suivante avec le même ensemble de clés

\begin{docCommand}{tcbuselibrary}{\marg{liste de clés}}
  Charge les bibliothèques énumérées dans la \meta{liste de clés}.
\begin{dispListing}
\tcbuselibrary{listings,theorems}
\end{dispListing}
\end{docCommand}


% The following keys are used inside |\tcbuselibrary| respectively
% |\usepackage| without the key tree path |/tcb/library/|.

Les clés suivantes peuvent être utilisées dans |\tcbuselibrary| (ou dans
|\usepackage|) sans avoir à saisir le chemin de l'arborescence des clés |/tcb/library/|.

% \begin{docTcbKey}[library]{skins}{}{\mylib{skins}}
%   Loads the package |tikz| \cite{tantau:tikz_and_pgf} and provides additional
%   styles (skins) for the appearance of the colored boxes; see
%   Section~\ref{sec:skins} from page~\pageref{sec:skins}.
% \end{docTcbKey}

\begin{docTcbKey}[library]{skins}{}{\mylib{skins}}
  Charge le paquet |tikz| \cite{tantau:tikz_and_pgf} et fournit des styles supplémentaires
  (skins) pour l'apparence des boîtes colorées~; consultez la
  section~\ref{sec:skins} à la page~\pageref{sec:skins}.
\end{docTcbKey}

% \begin{docTcbKey}[library]{vignette}{}{\mylib{vignette}}
%   Provides code for more ornamental; see
%   Section~\ref{sec:vignette} from page~\pageref{sec:vignette}.
% \end{docTcbKey}

\begin{docTcbKey}[library]{vignette}{}{\mylib{vignette}}
  Fournit du code pour plus d'ornements~; consultez la
  section~\ref{sec:vignette} à la page~\pageref{sec:vignette}.
\end{docTcbKey}

% \begin{docTcbKey}[library]{raster}{}{\mylib{raster}}
%   Provides additional macros and options for typesetting multiple
%   boxes arranged in a kind of raster;
%   see Section~\ref{sec:raster} from page~\pageref{sec:raster}.
% \end{docTcbKey}

\begin{docTcbKey}[library]{raster}{}{\mylib{raster}}
  Fournit des macros supplémentaires et des options pour saisir plusieurs boîtes
  arrangées sur une espèce de trame~; consultez la
  section~\ref{sec:raster} à la page~\pageref{sec:raster}.
\end{docTcbKey}

% \begin{docTcbKey}[library]{listings}{}{\mylib{listings}}
%   Loads the package |listings| \cite{hoffmann:listings} and provides additional
%   macros for typesetting listings which are described in Section~\ref{sec:listings}
%   from page~\pageref{sec:listings}.
% \end{docTcbKey}

\begin{docTcbKey}[library]{listings}{}{\mylib{listings}}
  Charge le paquet |listings| \cite{hoffmann:listings} et fournit des macros supplémentaires
  décrites à la section~\ref{sec:listings}
  à la page~\pageref{sec:listings} pour saisir des programmes informatiques (listings).
\end{docTcbKey}

% \begin{docTcbKey}[library]{listingsutf8}{}{\mylib{listingsutf8}}
%   Loads the packages |listings| \cite{hoffmann:listings} and
%   |listingsutf8| \cite{oberdiek:listingsutf8} for UTF-8 support.
%   This is a variant of the library \mylib{listings}
%   and is described in Section \ref{sec:listings}
%   from page~\pageref{sec:listings}.
% \end{docTcbKey}

\begin{docTcbKey}[library]{listingsutf8}{}{\mylib{listingsutf8}}
  Charge les paquets  |listings| \cite{hoffmann:listings} et
  |listingsutf8| \cite{oberdiek:listingsutf8} pour la prise en charge de l'UTF-8.
  Il s'agit d'une variante de la bibliothèque \mylib{listings}
  décrite à la section \ref{sec:listings} à la page~\pageref{sec:listings}.
\end{docTcbKey}

% \begin{docTcbKey}[library]{minted}{}{\mylib{minted}}
%   Loads the package |minted| \cite{poore:minted} to
%   typeset listings with the |Pygments| \cite{pygments:web} tool,
%   also see \Vref{sec:listings}.
% \end{docTcbKey}

\begin{docTcbKey}[library]{minted}{}{\mylib{minted}}
  Charge le paquet |minted| \cite{poore:minted} qui permet de saisir des programmes
  informatiques avec l'outils |Pygments| \cite{pygments:web},
  consultez également la section \Vref{sec:listings}.
\end{docTcbKey}

% \begin{docTcbKey}[library]{theorems}{}{\mylib{theorems}}
%   Provides additional
%   macros for typesetting theorems which are described in Section~\ref{sec:theorems}
%   à la page~\pageref{sec:theorems}.
% \end{docTcbKey}

\begin{docTcbKey}[library]{theorems}{}{\mylib{theorems}}
  Fournit des macros supplémentaires pour la saisie de
  théorèmes comme décrit dans la Section~\ref{sec:theorems}
  à la page~\pageref{sec:theorems}.
\end{docTcbKey}

% \begin{docTcbKey}[library]{breakable}{}{\mylib{breakable}}
%   Provides support for automatic box breaking from one page to another;
%   see \Fullref{sec:breakable}.
% \end{docTcbKey}

\begin{docTcbKey}[library]{breakable}{}{\mylib{breakable}}
  Fournit la prise en charge de la division automatique des boîtes d'une page à la suivante
  comme décrit \Fullref{sec:breakable}.
\end{docTcbKey}

% \begin{docTcbKey}[library]{magazine}{}{\mylib{magazine}}
%   Provides support for storing broken box parts to be used later or
%   in interchanged order, \Fullref{sec:magazine}.
% \end{docTcbKey}

\begin{docTcbKey}[library]{magazine}{}{\mylib{magazine}}
  Permet l'enregistrement des différentes parties des boîtes divisées afin de les
  réutiliser plus tard ou en modifiant leur ordre, \Fullref{sec:magazine}.
\end{docTcbKey}

% \begin{docTcbKey}[library]{poster}{}{\mylib{poster}}
%   Provides support for creating posters, \Fullref{sec:poster}.
% \end{docTcbKey}

\begin{docTcbKey}[library]{poster}{}{\mylib{poster}}
  Permet la création de posters, \Fullref{sec:poster}.
\end{docTcbKey}

% \begin{docTcbKey}[library]{fitting}{}{\mylib{fitting}}
%   Provides support for font size adaption of the box content to
%   the box dimensions;
%   see Section~\ref{sec:fitting} from page~\pageref{sec:fitting}.
% \end{docTcbKey}

\begin{docTcbKey}[library]{fitting}{}{\mylib{fitting}}
  Permet l'ajustement de la taille de la police du contenu de la boîte aux
  dimensions de la boîte~;
  Consultez la section~\ref{sec:fitting} à la page~\pageref{sec:fitting}.
\end{docTcbKey}

% \begin{docTcbKey}[library]{hooks}{}{\mylib{hooks}}
%   Extends several option keys to 'hookable' keys;
%   see Section~\ref{sec:hooks} from page~\pageref{sec:hooks}.
% \end{docTcbKey}

\begin{docTcbKey}[library]{hooks}{}{\mylib{hooks}}
  Rend plusieurs clés d'option «~extensibles~»~;
  Consultez la section~\ref{sec:hooks} à la page~\pageref{sec:hooks}.
\end{docTcbKey}

\clearpage
% \begin{docTcbKey}[library]{xparse}{}{\mylib{xparse}}
%   Provides document command production with |xparse| for |tcolorbox|;
%   see Section~\ref{sec:xparse} from page~\pageref{sec:xparse}.
% \end{docTcbKey}

\begin{docTcbKey}[library]{xparse}{}{\mylib{xparse}}
  Fournit des commandes de production de document avec |xparse| pour |tcolorbox|;
  consultez la section~\ref{sec:xparse} à la page~\pageref{sec:xparse}.
\end{docTcbKey}

% \begin{docTcbKey}[library]{external}{}{\mylib{external}}
%   Provides externalization support for stand-alone document snippets,
%   see \Fullref{sec:external}.
% \end{docTcbKey}

\begin{docTcbKey}[library]{external}{}{\mylib{external}}
  Permet la prise en charge de l'externalisation de fichiers indépendants contenants
  des extraits de code, consultez la section \Fullref{sec:external}.
\end{docTcbKey}

% \begin{docTcbKey}[library]{documentation}{}{\mylib{documentation}}
%   Provides additional macros for typesetting \LaTeX\ documentations
%   which are described in Section~\ref{sec:documentation}
%   from page~\pageref{sec:documentation}.
% \end{docTcbKey}

\begin{docTcbKey}[library]{documentation}{}{\mylib{documentation}}
  Fournit des macros supplémentaires pour la saisie de documentation \LaTeX\
  comme décrit dans la section~\ref{sec:documentation}
  à la page~\pageref{sec:documentation}.
\end{docTcbKey}

% \begin{docTcbKey}[library]{many}{}{style, no value}
%   Loads the libraries \mylib{skins}, \mylib{breakable}, \mylib{raster}, \mylib{hooks},
%   \mylib{theorems}, \mylib{fitting}, and \mylib{xparse}.
%   Use this shortcut, if you want to use all features of |tcolorbox|
%   with exception of typesetting listings and using
%   the specialized \mylib{documentation} library.
% \end{docTcbKey}

\begin{docTcbKey}[library]{many}{}{style, no value}
  Charge les bibliothèques \mylib{skins}, \mylib{breakable}, \mylib{raster}, \mylib{hooks},
  \mylib{theorems}, \mylib{fitting} et \mylib{xparse}.
  N'utilisez ce raccourci que si vous avez l'intention
  d'utiliser toutes les fonctionnalités de |tcolorbox| sauf la saisie de programme
  informatique et la bibliothèque spécialisée \mylib{documentation}.
\end{docTcbKey}

% \begin{docTcbKey}[library]{most}{}{style, no value}
%   Loads all libraries except \mylib{minted} and \mylib{documentation}.
%   Use this shortcut, if you want to use all features of |tcolorbox|
%   with exception of using the |minted| package and using
%   the specialized \mylib{documentation} library.
% \end{docTcbKey}

\begin{docTcbKey}[library]{most}{}{style, no value}
  Charge toutes les bibliothèques sauf \mylib{minted} et \mylib{documentation}.
  N'utilisez ce raccourci que si vous avez l'intention
  d'utiliser toutes les fonctionnalités de |tcolorbox|
  sauf le paquet |minted| et la bibliothèque spécialisée \mylib{documentation}.
\end{docTcbKey}

% \begin{docTcbKey}[library]{all}{}{style, no value}
%   Loads all libraries. Use this shortcut only, if you intend to use the
%   \mylib{documentation} library.
% \end{docTcbKey}

\begin{docTcbKey}[library]{all}{}{style, no value}
  Charge toutes les bibliothèques. N'utilisez ce raccourci que si vous avez l'intention
  d'utiliser la bibliothèque \mylib{documentation}.
\end{docTcbKey}


\begin{extcolorbox}[runs=2]{intro_packageoverview}
  [title={Paquets \texttt{tcolorbox}},center title,fonttitle=\bfseries,arc=0pt,
  colback=red!10!white,
  interior style={fill tile image*={width=2cm}{pink_marble.png},fill image opacity=0.5},
  colframe=red!50!black]
  \begin{tcolorbox}[beamer,adjusted title=Comportement de base,colframe=blue!50!black,colback=blue!10!white]
  Paquet de base
  \end{tcolorbox}
  \begin{tcbitemize}[raster columns=3,raster before skip=2mm,raster after skip=0pt,
    raster equal height,beamer,colframe=blue!50!black,colback=blue!10!white]
  \tcbitem[adjusted title=Fonctionnalités avancées]
    \mylib{breakable}\\
    \mylib{external}\\
    \mylib{fitting}\\
    \mylib{hooks}\\
    \mylib{magazine}\\
    \mylib{poster}\\
    \mylib{raster}\\
    \mylib{skins}\\
    \mylib{theorems}\\
    \mylib{vignette}\\
    \mylib{xparse}
  \tcbitem[adjusted title=Listings avancées]
    \mylib{listings}\\
    \mylib{listingsutf8}
    \tcblower
    \mylib{minted}
  \tcbitem[adjusted title=Documentation]
    \mylib{documentation}
  \end{tcbitemize}
\end{extcolorbox}

